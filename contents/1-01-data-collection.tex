\section{Модуль 1: Сбор данных для подготовки датасета по сети Bitcoin}

\subsection{Обзор задачи}

В рамках первого модуля НИРС была поставлена задача по сбору данных для подготовки датасета, предназначенного для обучения классификатора криптовалютных транзакций. Основной целью являлось создание структурированного набора данных, содержащего информацию об адресах Bitcoin и связанных с ними транзакциях.

\subsection{Источники данных}

\subsubsection{Датасет Гарвардского университета}

В качестве базового источника данных был использован датасет Гарвардского университета, доступный по адресу \url{https://dataverse.harvard.edu/dataset.xhtml?persistentId=doi:10.7910/DVN/KEWU0N}. Данный датасет содержит пары "адрес-категория" для Bitcoin-адресов, где каждая категория представляет собой тип сервиса или организации, связанной с адресом.

Датасет включает следующие категории адресов:
\begin{itemize}
    \item \textbf{exchange} --- биржи криптовалют
    \item \textbf{gambling} --- азартные игры
    \item \textbf{miner} --- майнеры
    \item \textbf{services} --- различные сервисы
    \item \textbf{coinjoin-like} --- адреса, связанные с технологиями повышения приватности
\end{itemize}

Общее количество адресов в исходном датасете составляет 8,810 записей.

\subsubsection{API сервис WalletExplorer.com}

Для получения дополнительной информации об адресах и транзакциях был использован API сервис \url{https://www.walletexplorer.com/api}. Данный сервис предоставляет JSON API для получения:

\begin{itemize}
    \item Информации об адресах (название кошелька, количество транзакций, последний обновленный блок)
    \item Списка транзакций для каждого адреса (до 100 последних транзакций)
    \item Детальной информации о транзакциях (входы, выходы, размер, время)
\end{itemize}

\subsection{Архитектура системы сбора данных}

\subsubsection{Основные компоненты}

Система сбора данных реализована на языке Python и включает следующие основные компоненты:

\begin{enumerate}
    \item \textbf{WalletExplorerAPI} --- класс для взаимодействия с API сервиса
    \item \textbf{DataPersistence} --- класс для сохранения и загрузки данных
    \item \textbf{BitcoinDataCollector} --- основной класс, координирующий процесс сбора
    \item \textbf{CollectorConfig} --- конфигурация системы
\end{enumerate}

\subsubsection{Структуры данных}

Система использует две основные структуры данных, реализованные как Python dataclass:

\begin{itemize}
    \item \textbf{AddressData} --- содержит информацию об адресе:
    \begin{itemize}
        \item \texttt{address} --- Bitcoin адрес (строка)
        \item \texttt{wallet\_name} --- название кошелька/сервиса
        \item \texttt{wallet\_id} --- уникальный идентификатор кошелька
        \item \texttt{found} --- флаг наличия адреса в базе WalletExplorer
        \item \texttt{txs\_count} --- общее количество транзакций для адреса
        \item \texttt{updated\_to\_block} --- номер последнего обработанного блока
        \item \texttt{transactions} --- список связанных транзакций (до 100 последних)
    \end{itemize}
    
    \item \textbf{TransactionData} --- содержит информацию о транзакции:
    \begin{itemize}
        \item \texttt{address} --- связанный Bitcoin адрес
        \item \texttt{txid} --- уникальный идентификатор транзакции
        \item \texttt{amount\_sent} --- сумма отправленных средств (в BTC)
        \item \texttt{amount\_received} --- сумма полученных средств (в BTC)
        \item \texttt{block\_height} --- высота блока, содержащего транзакцию
        \item \texttt{block\_pos} --- позиция транзакции в блоке
        \item \texttt{time} --- временная метка транзакции (Unix timestamp)
        \item \texttt{balance} --- баланс адреса после транзакции
        \item \texttt{used\_as\_input} --- флаг использования адреса как входа
        \item \texttt{used\_as\_output} --- флаг использования адреса как выхода
        \item \texttt{found} --- флаг успешного получения данных
        \item \texttt{label} --- метка кошелька/сервиса
        \item \texttt{wallet\_id} --- идентификатор кошелька
        \item \texttt{size} --- размер транзакции в байтах
        \item \texttt{is\_coinbase} --- флаг coinbase транзакции
        \item \texttt{inputs\_count} --- количество входов транзакции
        \item \texttt{outputs\_count} --- количество выходов транзакции
        \item \texttt{inputs} --- детальная информация о входах (JSON)
        \item \texttt{outputs} --- детальная информация о выходах (JSON)
    \end{itemize}
\end{itemize}

\subsection{Процесс сбора данных}

\subsubsection{Этапы сбора}

Процесс сбора данных состоит из следующих этапов:

\begin{enumerate}
    \item \textbf{Загрузка исходных адресов} --- чтение адресов из датасета Гарварда
    \item \textbf{Получение информации об адресах} --- запрос к API для каждого адреса
    \item \textbf{Сбор транзакций} --- получение до 100 последних транзакций для каждого адреса
    \item \textbf{Детализация транзакций} --- получение подробной информации о каждой транзакции
    \item \textbf{Сохранение данных} --- запись результатов в CSV файлы
\end{enumerate}

\subsubsection{Особенности реализации}

Система включает следующие важные особенности:

\begin{itemize}
    \item \textbf{Инкрементальный сбор} --- возможность продолжения сбора с места остановки:
    \begin{itemize}
        \item Автоматическое определение уже обработанных адресов и транзакций
        \item Загрузка состояния из существующих CSV файлов при запуске
        \item Пропуск дублирующихся запросов для экономии времени
    \end{itemize}
    
    \item \textbf{Обработка ошибок} --- устойчивость к сетевым сбоям и ограничениям API:
    \begin{itemize}
        \item Автоматические повторные попытки при сетевых ошибках (до 3 попыток)
        \item Обработка HTTP 429 (Rate Limiting) с увеличением задержки
        \item Продолжение работы при ошибках отдельных запросов
        \item Подробное логирование всех ошибок и предупреждений
    \end{itemize}
    
    \item \textbf{Параллельная обработка} --- возможность использования нескольких потоков:
    \begin{itemize}
        \item Настраиваемое количество worker-потоков (по умолчанию 5)
        \item ThreadPoolExecutor для эффективного управления потоками
        \item Ускорение сбора данных в 3-5 раз по сравнению с последовательной обработкой
    \end{itemize}
    
    \item \textbf{Контроль скорости} --- настраиваемые задержки между запросами:
    \begin{itemize}
        \item Базовая задержка 1.0 секунда между запросами
        \item Автоматическое увеличение задержки при получении HTTP 429
        \item Возможность настройки задержки через параметры командной строки
    \end{itemize}
    
    \item \textbf{Пакетное сохранение} --- периодическое сохранение данных:
    \begin{itemize}
        \item Сохранение адресов пакетами по 20 записей
        \item Сохранение транзакций пакетами по 100 записей
        \item Атомарные операции записи для предотвращения потери данных
        \item Поддержка режима append для продолжения сбора
    \end{itemize}
    
    \item \textbf{Двухэтапный сбор данных}:
    \begin{itemize}
        \item Первый этап: получение информации об адресах и списка транзакций
        \item Второй этап: детализация каждой транзакции с полной информацией
        \item Оптимизация запросов для минимизации нагрузки на API
    \end{itemize}
\end{itemize}

\subsection{Результаты сбора данных}

\subsubsection{Статистика собранных данных}

В результате работы системы сбора данных были получены следующие результаты:

\begin{itemize}
    \item \textbf{Обработано адресов}: 8,810 (100\% от исходного датасета)
    \item \textbf{Найдено в базе WalletExplorer}: значительная часть адресов
    \item \textbf{Собрано транзакций}: 292,820 транзакций
    \item \textbf{Время сбора}: несколько часов с учетом ограничений API
\end{itemize}

\subsubsection{Выходные файлы}

Система создает следующие выходные файлы:

\begin{enumerate}
    \item \textbf{addresses.csv} --- основная информация об адресах и их транзакциях:
    \begin{itemize}
        \item Содержит 8,810 записей об адресах
        \item Включает встроенные JSON-данные о транзакциях для каждого адреса
        \item Размер файла: несколько мегабайт
    \end{itemize}
    
    \item \textbf{transactions.csv} --- детальная информация о всех транзакциях:
    \begin{itemize}
        \item Содержит 292,820 записей о транзакциях
        \item Полная информация о входах и выходах каждой транзакции
        \item Размер файла: 4,62 ГБ
    \end{itemize}
    
    \item \textbf{bitcoin\_collector.log} --- лог-файл с информацией о процессе сбора:
    \begin{itemize}
        \item Подробная информация о ходе выполнения
        \item Статистика обработанных адресов и транзакций
        \item Записи об ошибках и предупреждениях
    \end{itemize}
\end{enumerate}

\subsubsection{Облачное хранилище данных}

Все собранные данные доступны в облачном хранилище Yandex Disk по ссылке: \url{https://disk.yandex.ru/d/FN1jFpsPemgDhA}

В хранилище содержатся CSV файлы:
\begin{itemize}
    \item Исходный датасет Harvard с адресами и категориями
    \item Обработанные данные адресов (addresses.csv)
    \item Детальная информация о транзакциях (transactions.csv, 4,62 ГБ)
\end{itemize}

\subsection{Технические характеристики}

\subsubsection{Конфигурация системы}

Система поддерживает настройку следующих параметров:

\begin{itemize}
    \item Задержка между запросами (по умолчанию: 1.0 секунда)
    \item Количество параллельных потоков (по умолчанию: 5)
    \item Размер пакета для сохранения адресов (по умолчанию: 20)
    \item Размер пакета для сохранения транзакций (по умолчанию: 100)
    \item Таймаут запросов (по умолчанию: 30 секунд)
    \item Количество повторных попыток (по умолчанию: 3)
\end{itemize}

\subsubsection{Обработка ошибок}

Система включает комплексную обработку ошибок:

\begin{itemize}
    \item Автоматические повторные попытки при сетевых ошибках
    \item Обработка ограничений скорости API (HTTP 429)
    \item Логирование всех ошибок и предупреждений
    \item Продолжение работы при ошибках отдельных запросов
\end{itemize}

\subsection{Заключение по модулю}

Первый модуль НИРС успешно завершен. Была создана надежная система сбора данных, которая позволила получить структурированный датасет, содержащий:

\begin{itemize}
    \item Информацию о 8,810 Bitcoin-адресах с категоризацией
    \item Детальные данные о десятках тысяч связанных транзакций
    \item Метаданные о кошельках и сервисах
    \item Временные характеристики транзакций
\end{itemize}

Полученный датасет готов для использования в следующих модулях НИРС для обучения классификатора криптовалютных транзакций. Система сбора данных может быть легко адаптирована для сбора дополнительных данных или работы с другими источниками.
