\subsection{Обзор доступных Bitcoin-датасетов}

В настоящее время в открытом доступе существует несколько крупных датасетов, содержащих данные о Bitcoin-транзакциях и адресах. Каждый из этих датасетов имеет свою специфику и структуру данных.

\subsubsection{Датасеты, структурированные по транзакциям}

\textbf{Elliptic Dataset} — один из наиболее известных датасетов для анализа Bitcoin-транзакций. Содержит 200,000 транзакций с графовой структурой, где каждая транзакция представлена 167 признаками, включая локальные и агрегированные графовые характеристики. Датасет размечен на три класса: легитимные транзакции (21\%), подозрительные (2\%) и неразмеченные (77\%). Временной период охватывает 49 временных интервалов.

\subsubsection{Датасеты, структурированные по адресам}

\textbf{Bitcoin Heist Dataset} содержит 2,916,697 записей, структурированных по Bitcoin-адресам. Каждый адрес характеризуется 9 признаками: количество раундов смешивания, вес транзакций, количество связанных транзакций, цикличность операций и другие топологические характеристики. Датасет включает 29 семейств ransomware-программ и легитимные адреса.

\textbf{Real-CATS Dataset} содержит 209,399 Bitcoin-адресов, разделенных на криминальные (103,203) и биржевые (106,196) адреса. Датасет основан на реальных криминальных делах и предоставляет метаданные о типах преступлений.

\textbf{BABD Dataset} содержит 544,462 записи о поведении Bitcoin-адресов с 148 признаками. Датасет классифицирует адреса по 13 типам: биржи, майнинг-пулы, обычные пользователи и другие категории.
