\conclusion
% Disable hyphenation locally within conclusion
\begingroup\hyphenpenalty=10000\exhyphenpenalty=10000\sloppy
В рамках данной научно-исследовательской работы разработан метод создания единого адресного датасета Bitcoin с признаками, извлечёнными через Bitcoin-эксплорер, и экспертной разметкой из существующих источников; проведён анализ доступных датасетов и выявлены их ключевые характеристики; разработан метод извлечения адресов и сбора их признаков; предложена система унификации разметки на основе экспертных данных и методология разрешения конфликтов при пересечении адресов; обоснован подход создания единого адресного датасета как наиболее практичного решения. Созданная система сбора данных показала высокую эффективность, собрав 8\,810 Bitcoin-адресов с 292\,820 транзакциями объёмом 4.62~ГБ; использование многопоточности и системы обработки ошибок обеспечило надёжность процесса. Применение датасета Гарвардского университета в качестве базового источника обеспечило высокое качество экспертной разметки, критически важное для корректности последующего анализа; архитектура решения масштабируема и позволяет адаптировать процесс для других криптовалют и временных окон. Из 34 извлечённых признаков наиболее важными для классификации оказались временные (\texttt{transaction\_frequency}) и экономические (\texttt{avg\_amount\_sent}) характеристики, что подтверждает значимость поведенческих паттернов; выявлена существенная несбалансированность классов (коэффициент 39.85), где майнеры составляют 43.7\% от числа адресов, что требует специальных подходов при обучении; обнаружены 28 значимых корреляций (включая полные, r=1.000), что указывает на возможность снижения размерности без потери информативности. Созданный датасет готов к применению методов машинного обучения с рекомендацией ансамблевых моделей и техник балансировки; выявленные паттерны поведения (майнеры, биржи, азартные игры) имеют чёткую экономическую интерпретацию, повышая практическую ценность исследования; этапы работы документированы и автоматизированы, что обеспечивает воспроизводимость и дальнейшее развитие методологии. Результаты представляют практическую значимость для исследователей анализа Bitcoin-транзакций, разработчиков систем детектирования мошенничества, аналитиков блокчейн-безопасности и правоохранительных органов. Перспективные направления включают практическую реализацию метода создания единого адресного датасета, дальнейшую разработку алгоритмов сбора признаков через Bitcoin-эксплорер, создание системы валидации качества разметки и разрешения конфликтов, развитие методов машинного обучения для классификации адресов, интеграцию дополнительных источников экспертной разметки и построение системы мониторинга новых мошеннических адресов. Исходный код и данные, реализующие предложенные методы, а также результаты экспериментов доступны в открытом доступе в репозитории \cite{github_repo}, содержащем скрипты для сбора данных, код извлечения признаков, Jupyter-ноутбуки с анализом и визуализацией, обработанные датасеты с метаданными и сопроводительную документацию.
\par\endgroup
