\conclusion

В рамках данной научно-исследовательской работы был разработан метод создания единого адресного датасета Bitcoin с признаками, извлеченными через Bitcoin-эксплорер, и экспертной разметкой из существующих источников.

\subsection{Достигнутые результаты}

\begin{enumerate}
    \item Проведен анализ существующих Bitcoin-датасетов и выявлены их ключевые характеристики
    \item Разработан метод извлечения адресов и сбора их признаков через Bitcoin-эксплорер
    \item Предложена система унификации разметки на основе экспертных данных
    \item Создана методология разрешения конфликтов при пересечении адресов в разных датасетах
    \item Обоснован подход создания единого адресного датасета как наиболее практичного решения
\end{enumerate}

\subsection{Основные выводы исследования}

\textbf{Выводы по сбору данных:}

1. \textbf{Эффективность разработанной системы}: Созданная система сбора данных показала высокую эффективность, собрав 8,810 Bitcoin-адресов с 292,820 транзакциями объемом 4.62 ГБ. Использование многопоточности и системы обработки ошибок обеспечило надежность процесса сбора.

2. \textbf{Качество исходных данных}: Применение датасета Гарвардского университета в качестве базового источника обеспечило высокое качество экспертной разметки адресов, что критически важно для корректности последующего анализа.

3. \textbf{Масштабируемость решения}: Разработанная архитектура системы позволяет легко адаптировать процесс сбора для других криптовалют или расширения временного охвата данных.

\textbf{Выводы по анализу данных:}

1. \textbf{Информативность признаков}: Из 34 извлеченных признаков наиболее важными для классификации оказались временные (transaction\_frequency) и экономические (avg\_amount\_sent) характеристики, что подтверждает гипотезу о значимости поведенческих паттернов.

2. \textbf{Несбалансированность классов}: Выявлена значительная несбалансированность датасета (коэффициент 39.85), где майнеры составляют 43.7\% от общего количества адресов, что требует специальных подходов при обучении классификатора.

3. \textbf{Корреляционная структура}: Обнаружены 28 значимых корреляций между признаками, включая полные корреляции (r = 1.000), что указывает на возможность снижения размерности признакового пространства без потери информативности.

\textbf{Выводы по практическому применению:}

1. \textbf{Готовность к машинному обучению}: Созданный датасет полностью готов для применения методов машинного обучения с рекомендацией использования ансамблевых методов (Random Forest, Gradient Boosting) и техник балансировки классов.

2. \textbf{Интерпретируемость результатов}: Выявленные паттерны поведения различных типов кошельков (майнеры, биржи, азартные игры) имеют четкую экономическую интерпретацию, что повышает практическую ценность исследования.

3. \textbf{Воспроизводимость исследования}: Все этапы работы документированы и автоматизированы, что обеспечивает возможность воспроизведения результатов и дальнейшего развития методологии.

\subsection{Практическая значимость}

Результаты исследования имеют высокую практическую значимость для:

\begin{itemize}
    \item Исследователей в области анализа Bitcoin-транзакций
    \item Разработчиков систем обнаружения мошенничества в криптовалютах
    \item Аналитиков блокчейн-безопасности
    \item Правоохранительных органов для расследования преступлений с Bitcoin
\end{itemize}

\subsection{Направления дальнейших исследований}

Перспективными направлениями дальнейших исследований являются:

\begin{itemize}
    \item Практическая реализация метода создания единого адресного датасета
    \item Разработка алгоритмов сбора признаков адресов через Bitcoin-эксплорер
    \item Создание системы валидации качества разметки и разрешения конфликтов
    \item Разработка методов машинного обучения для классификации адресов
    \item Интеграция дополнительных источников экспертной разметки
    \item Создание системы мониторинга новых мошеннических адресов
\end{itemize}

\subsection{Исходный код и данные}

Полный исходный код, реализующий предложенные методы сбора и обработки данных, а также результаты экспериментов доступны в открытом доступе в GitHub репозитории \cite{github_repo}. Репозиторий содержит:

\begin{itemize}
    \item Скрипты для сбора данных через Bitcoin-эксплорер
    \item Код для извлечения признаков из транзакционных данных
    \item Jupyter Notebook с анализом и визуализацией результатов
    \item Обработанные датасеты и метаданные
    \item Документацию по использованию инструментов
\end{itemize}
