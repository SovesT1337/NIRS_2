\section{Модуль 2: Обработка и анализ Bitcoin-данных}

\subsection{Связь с первым модулем}

Второй модуль НИРС является логическим продолжением первого модуля и направлен на обработку и анализ данных, собранных в рамках первого этапа работы. 

\textbf{Входные данные} для второго модуля получены в результате выполнения первого модуля:
\begin{itemize}
    \item Обработка CSV файлов \texttt{addresses.csv} и \texttt{transactions.csv}
    \item Извлечение 34 новых признаков из сырых данных транзакций
    \item Создание структурированного датасета для обучения классификатора
\end{itemize}

\textbf{Цель второго модуля}: преобразование сырых данных транзакций в набор признаков, пригодных для машинного обучения и классификации типов Bitcoin-адресов.

\subsection{Методология сбора и обработки признаков}

\subsubsection{Входные данные}

Входными данными для второго модуля НИРС послужили CSV файлы, полученные в результате работы первого модуля:

\begin{itemize}
    \item \textbf{addresses.csv} - основная информация об 8,809 Bitcoin-адресах с категоризацией
    \item \textbf{transactions.csv} - детальная информация о 292,819 транзакциях (размер 4,62 ГБ)
    \item \textbf{addresses-labels.csv} - метки классов для адресов из датасета Гарвардского университета
\end{itemize}

Данные были собраны с использованием API сервиса WalletExplorer.com и содержат:
\begin{itemize}
    \item Информацию о Bitcoin-адресах и их метках (exchange, gambling, miner, services, coinjoin-like, mining\_pool)
    \item Детальные данные о транзакциях (входы, выходы, суммы, временные метки)
    \item Структурные характеристики транзакций (размер, количество входов/выходов, coinbase флаги)
    \item Метаданные о кошельках и сервисах
\end{itemize}

\subsubsection{Архитектура извлечения признаков}

Процесс извлечения признаков реализован в классе \texttt{BitcoinTransactionAnalyzer} и включает следующие этапы:

\begin{enumerate}
    \item \textbf{Загрузка данных}: объединение данных об адресах, транзакциях и метках
    \item \textbf{Анализ транзакций}: извлечение признаков для каждого адреса
    \item \textbf{Агрегация}: вычисление статистических характеристик
    \item \textbf{Сохранение}: экспорт в CSV формат с инкрементальным сохранением
\end{enumerate}

\subsection{Классификация и описание признаков}

В рамках второго модуля НИРС был проведен анализ датасета Bitcoin-адресов, содержащего 8,115 записей с 34 признаками. Признаки разделены на следующие категории:

\subsubsection{Структурные признаки транзакций}

Структурные признаки характеризуют архитектуру транзакций Bitcoin и отражают паттерны использования UTXO модели. Данные признаки вычисляются на основе анализа всех транзакций, связанных с конкретным адресом.

\paragraph{Количественные характеристики входов и выходов}

Для каждого адреса вычисляются следующие статистические показатели:

\begin{itemize}
    \item \texttt{avg\_inputs} - среднее арифметическое количество входов:
    \begin{equation}
        \text{avg\_inputs} = \frac{1}{n} \sum_{i=1}^{n} \text{inputs\_count}_i
    \end{equation}
    где $n$ - общее количество транзакций адреса, $\text{inputs\_count}_i$ - количество входов в $i$-й транзакции.
    
    \item \texttt{median\_inputs} - медианное значение количества входов, характеризующее типичное поведение адреса.
    
    \item \texttt{avg\_outputs, median\_outputs} - аналогично для выходов транзакций.
    
    \item \texttt{max\_inputs, max\_outputs, min\_inputs, min\_outputs} - экстремальные значения, показывающие диапазон вариативности.
\end{itemize}

\paragraph{Отношение входов к выходам}

\texttt{inputs\_outputs\_ratio} вычисляется как:
\begin{equation}
    \text{inputs\_outputs\_ratio} = \frac{\text{avg\_inputs}}{\text{avg\_outputs}}
\end{equation}

Этот показатель отражает стратегию управления UTXO:
\begin{itemize}
    \item Значения > 1 указывают на консолидацию множественных входов
    \item Значения < 1 характерны для адресов, создающих множество выходов
    \item Значения ≈ 1 типичны для простых транзакций
\end{itemize}

\paragraph{Размер транзакций}

\texttt{avg\_tx\_size, median\_tx\_size} - размер транзакций в байтах, включающий:
\begin{itemize}
    \item Заголовок транзакции (4 байта)
    \item Количество входов и выходов
    \item Скрипты входов и выходов
    \item Блокировка времени
\end{itemize}

\textbf{Научная интерпретация}: 
\begin{itemize}
    \item \textbf{Биржи} демонстрируют высокие значения \texttt{avg\_inputs} и \texttt{avg\_outputs} вследствие необходимости агрегации множественных депозитов и создания множественных выплат.
    \item \textbf{Майнеры} характеризуются низкими значениями \texttt{avg\_inputs} (обычно 1) и \texttt{avg\_outputs} (1-2), что отражает простоту coinbase транзакций.
    \item \textbf{Микшинг-сервисы} показывают высокие значения \texttt{inputs\_outputs\_ratio} из-за консолидации множественных входов для обеспечения анонимности.
\end{itemize}

\subsubsection{UTXO паттерны}

UTXO (Unspent Transaction Output) паттерны отражают специфические особенности модели Bitcoin и стратегии управления непотраченными выходами транзакций.

\paragraph{Пылевые транзакции}

\texttt{dust\_ratio} - доля пылевых транзакций, вычисляется как:
\begin{equation}
    \text{dust\_ratio} = \frac{\text{count}(\text{amount} < \text{DUST\_THRESHOLD})}{\text{total\_transactions}}
\end{equation}

где $\text{DUST\_THRESHOLD} = 546$ satoshi (0.00000546 BTC) - минимальная экономически целесообразная сумма.

\textbf{Алгоритм вычисления}:
\begin{enumerate}
    \item Для каждой транзакции проверяется сумма отправленных и полученных средств
    \item Подсчитывается количество транзакций с суммой < 546 satoshi
    \item Вычисляется отношение к общему количеству транзакций
\end{enumerate}

\paragraph{Паттерны входов}

\texttt{largest\_input\_ratio} - доля транзакций с одним крупным входом:
\begin{equation}
    \text{largest\_input\_ratio} = \frac{\text{count}(\text{inputs\_count} = 1 \land \text{amount\_sent} > 0)}{\text{total\_transactions}}
\end{equation}

\texttt{many\_small\_inputs\_ratio} - доля транзакций с множественными мелкими входами:
\begin{equation}
    \text{many\_small\_inputs\_ratio} = \frac{\text{count}(\text{inputs\_count} > 5 \land \text{amount\_sent} < 0.01 \text{ BTC})}{\text{total\_transactions}}
\end{equation}

\paragraph{Сдача и консолидация}

\texttt{change\_output\_ratio} - доля транзакций с сдачей, определяемая по наличию выхода на тот же адрес.

\textbf{Научная интерпретация}:
\begin{itemize}
    \item \textbf{Высокий dust\_ratio} характерен для спам-атак и тестовых транзакций
    \item \textbf{Высокий largest\_input\_ratio} указывает на консолидацию средств
    \item \textbf{Высокий many\_small\_inputs\_ratio} типичен для микшинг-сервисов и пулов майнинга
    \item \textbf{Низкий change\_output\_ratio} может указывать на автоматизированные системы
\end{itemize}

\subsubsection{Временные паттерны}

Временные паттерны характеризуют поведенческие особенности адресов во времени и позволяют выявить автоматизированные системы, атаки и типичные пользовательские паттерны.

\paragraph{Частота транзакций}

\texttt{transaction\_frequency} - средняя частота транзакций, вычисляется как:
\begin{equation}
    \text{transaction\_frequency} = \frac{\text{total\_transactions}}{\text{time\_span\_days}}
\end{equation}

где $\text{time\_span\_days} = \frac{\text{last\_tx\_time} - \text{first\_tx\_time}}{86400}$ - период активности в днях.

\paragraph{Всплески активности}

\texttt{burst\_activity\_ratio} - доля транзакций в всплесках активности, вычисляется по алгоритму:

\begin{enumerate}
    \item Группировка транзакций по временным окнам размером 24 часа
    \item Подсчет окон с количеством транзакций > 3
    \item Вычисление доли транзакций в таких окнах:
\end{enumerate}

\begin{equation}
    \text{burst\_activity\_ratio} = \frac{\sum_{w \in \text{burst\_windows}} \text{tx\_count}_w}{\text{total\_transactions}}
\end{equation}

где $\text{burst\_windows} = \{w : \text{tx\_count}_w > 3\}$ - окна с высокой активностью.

\paragraph{Период активности}

\texttt{time\_span\_days} - общий период активности адреса в днях, характеризующий долгосрочность использования.

\textbf{Научная интерпретация}:
\begin{itemize}
    \item \textbf{Высокая transaction\_frequency} (> 1 транзакция/день) характерна для бирж и автоматизированных систем
    \item \textbf{Высокий burst\_activity\_ratio} может указывать на:
        \begin{itemize}
            \item DDoS атаки на сеть Bitcoin
            \item Автоматизированные торговые боты
            \item Микшинг-сервисы в периоды высокой нагрузки
        \end{itemize}
    \item \textbf{Короткий time\_span\_days} может указывать на временные адреса или тестовые операции
\end{itemize}

\subsubsection{Экономические паттерны}

Экономические паттерны отражают финансовое поведение адресов и позволяют выявить типичные стратегии управления средствами различных категорий пользователей.

\paragraph{Статистические характеристики сумм}

Для каждого адреса вычисляются следующие показатели:

\begin{itemize}
    \item \texttt{avg\_amount\_sent} - среднее арифметическое отправленных сумм:
    \begin{equation}
        \text{avg\_amount\_sent} = \frac{1}{n} \sum_{i=1}^{n} \text{amount\_sent}_i
    \end{equation}
    
    \item \texttt{median\_amount\_sent} - медианное значение, более устойчивое к выбросам
    
    \item \texttt{avg\_amount\_received, median\_amount\_received} - аналогично для полученных сумм
    
    \item \texttt{max\_single\_tx, min\_single\_tx} - экстремальные значения, показывающие диапазон операций
\end{itemize}

\paragraph{Паттерны круглых сумм}

\texttt{round\_number\_ratio} - доля "круглых" сумм, вычисляется как:
\begin{equation}
    \text{round\_number\_ratio} = \frac{\text{count}(\text{amount} \in \text{ROUND\_AMOUNTS})}{\text{total\_amounts}}
\end{equation}

где $\text{ROUND\_AMOUNTS} = \{0.001, 0.01, 0.1, 1.0, 10.0, 100.0, 1000.0\}$ BTC.

\textbf{Алгоритм определения круглых сумм}:
\begin{enumerate}
    \item Проверка точного совпадения с предопределенными значениями
    \item Учет погрешности округления для больших сумм
    \item Нормализация по общему количеству транзакций
\end{enumerate}

\paragraph{Энтропия распределения сумм}

\texttt{value\_entropy} - энтропия Шеннона распределения сумм:
\begin{equation}
    H = -\sum_{i=1}^{k} p_i \log_2 p_i
\end{equation}

где $p_i$ - вероятность суммы в $i$-м интервале после дискретизации.

\textbf{Алгоритм вычисления энтропии}:
\begin{enumerate}
    \item Дискретизация сумм на интервалы
    \item Вычисление частот для каждого интервала
    \item Применение формулы Шеннона
\end{enumerate}

\textbf{Научная интерпретация}:
\begin{itemize}
    \item \textbf{Высокий round\_number\_ratio} характерен для:
        \begin{itemize}
            \item Бирж (стандартные суммы для торговли)
            \item Сервисов с фиксированными тарифами
            \item Автоматизированных систем
        \end{itemize}
    \item \textbf{Высокая value\_entropy} указывает на:
        \begin{itemize}
            \item Микшинг-сервисы (случайные суммы для анонимности)
            \item Сложные финансовые операции
            \item Естественное пользовательское поведение
        \end{itemize}
    \item \textbf{Низкая value\_entropy} типична для:
        \begin{itemize}
            \item Автоматизированных систем с фиксированными суммами
            \item Простых пользовательских операций
        \end{itemize}
\end{itemize}

\subsubsection{Сетевые паттерны}

Сетевые паттерны характеризуют взаимодействие адреса с другими участниками сети Bitcoin и отражают стратегии обеспечения анонимности и приватности.

\paragraph{Уникальность адресов}

\texttt{unique\_input\_addresses} и \texttt{unique\_output\_addresses} - количество уникальных адресов, с которыми взаимодействует анализируемый адрес.

\textbf{Алгоритм вычисления}:
\begin{enumerate}
    \item Извлечение всех адресов из входов и выходов транзакций
    \item Удаление дубликатов
    \item Подсчет уникальных адресов
\end{enumerate}

\paragraph{Коэффициент повторного использования}

\texttt{address\_reuse\_ratio} - коэффициент повторного использования адресов:
\begin{equation}
    \text{address\_reuse\_ratio} = \frac{\text{total\_interactions}}{\text{unique\_addresses}}
\end{equation}

где $\text{total\_interactions} = \text{unique\_input\_addresses} + \text{unique\_output\_addresses}$.

\paragraph{Коэффициент кластеризации}

\texttt{clustering\_coefficient} - мера локальной связности в сети транзакций, вычисляется как:
\begin{equation}
    C = \frac{2 \times \text{actual\_connections}}{\text{possible\_connections}}
\end{equation}

\textbf{Научная интерпретация}:
\begin{itemize}
    \item \textbf{Низкий address\_reuse\_ratio} характерен для:
        \begin{itemize}
            \item Приватных кошельков (стратегия "один адрес - одна транзакция")
            \item Анонимных сервисов
            \item Продвинутых пользователей
        \end{itemize}
    \item \textbf{Высокий address\_reuse\_ratio} типичен для:
        \begin{itemize}
            \item Бирж (повторное использование адресов для депозитов)
            \item Простых пользователей
            \item Автоматизированных систем
        \end{itemize}
    \item \textbf{Высокий clustering\_coefficient} указывает на:
        \begin{itemize}
            \item Централизованные системы
            \item Хаб-адреса в сети
            \item Координированную активность
        \end{itemize}
\end{itemize}

\subsubsection{Bitcoin-специфичные признаки}

Признаки, уникальные для протокола Bitcoin:

\begin{itemize}
    \item \texttt{coinbase\_ratio} - доля coinbase транзакций (характерно для майнеров)
    \item \texttt{multisig\_ratio} - доля мультиподписных транзакций
    \item \texttt{op\_return\_usage} - использование OP\_RETURN (для метаданных)
\end{itemize}

\subsection{Алгоритмы обработки признаков}

Все алгоритмы извлечения признаков реализованы в классе \texttt{BitcoinTransactionAnalyzer} (см. приложение А). Ниже представлены ключевые алгоритмы:

\subsubsection{Вычисление временных паттернов}

Для вычисления \texttt{burst\_activity\_ratio} используется алгоритм:
\begin{enumerate}
    \item Группировка транзакций по временным окнам (24 часа)
    \item Подсчет окон с количеством транзакций > 3
    \item Вычисление доли транзакций в таких окнах
\end{enumerate}

\textbf{Реализация}: метод \texttt{\_calculate\_burst\_activity()} в классе \texttt{BitcoinTransactionAnalyzer}.

\subsubsection{Определение круглых сумм}

Алгоритм проверки круглых сумм использует предопределенный набор значений:
\texttt{[0.001, 0.01, 0.1, 1.0, 10.0, 100.0, 1000.0]} BTC

\textbf{Реализация}: метод \texttt{\_is\_round\_number()} в классе \texttt{BitcoinTransactionAnalyzer}.

\subsubsection{Вычисление энтропии}

Энтропия сумм вычисляется по формуле Шеннона для дискретизированных значений сумм, что позволяет оценить разнообразие транзакционных сумм.

\textbf{Реализация}: метод \texttt{\_calculate\_entropy()} в классе \texttt{BitcoinTransactionAnalyzer}.

\subsubsection{Оптимизация производительности}

Для обработки больших объемов данных использованы:
\begin{itemize}
    \item Векторизованные операции NumPy
    \item Инкрементальное сохранение результатов
    \item Пакетная обработка адресов
\end{itemize}

\subsection{Распределение классов}

Датасет содержит 6 классов Bitcoin-адресов:

\begin{table}[h]
\centering
\begin{tabular}{|l|c|c|}
\hline
\textbf{Класс} & \textbf{Количество} & \textbf{Доля (\%)} \\
\hline
miner & 3,547 & 43.7 \\
exchange & 1,532 & 18.9 \\
services & 1,249 & 15.4 \\
gambling & 911 & 11.2 \\
coinjoin-like & 787 & 9.7 \\
mining\_pool & 89 & 1.1 \\
\hline
\textbf{Всего} & \textbf{8,115} & \textbf{100.0} \\
\hline
\end{tabular}
\caption{Распределение классов в датасете}
\label{tab:class_distribution}
\end{table}

\subsection{Анализ признаков}

\subsubsection{Структурные характеристики}

Анализ структурных признаков показал значительные различия между классами:

\begin{itemize}
    \item \textbf{Количество транзакций}: майнеры демонстрируют наибольшую активность (высокое значение \texttt{txs\_count})
    \item \textbf{Средние входы/выходы}: биржи характеризуются большим количеством входов и выходов в транзакциях
    \item \textbf{Размер транзакций}: майнинг-пулы имеют наиболее крупные транзакции
\end{itemize}

\subsubsection{Экономические паттерны}

Экономические признаки выявили следующие закономерности:

\begin{itemize}
    \item \textbf{Средние суммы}: биржи обрабатывают наибольшие объемы средств
    \item \textbf{Медианные значения}: показывают типичное поведение для каждого класса
    \item \textbf{Общие объемы}: майнеры и биржи лидируют по общим объемам транзакций
\end{itemize}

\subsection{Корреляционный анализ}

Корреляционный анализ основных признаков выявил следующие взаимосвязи:

\begin{itemize}
    \item Сильная положительная корреляция между количеством транзакций и общими объемами
    \item Корреляция между структурными признаками (входы/выходы) и размерами транзакций
    \item Связь между временными паттернами и экономическими характеристиками
\end{itemize}

\subsection{Статистический анализ и визуализация данных}

\subsubsection{Распределение классов в датасете}

Анализ распределения классов показал значительную несбалансированность датасета (таблица \ref{tab:class_distribution}). Доминирующий класс \texttt{miner} составляет 43.7\% от общего количества записей, что может привести к смещению модели классификации.

\begin{table}[h]
\centering
\begin{tabular}{|l|c|c|c|}
\hline
\textbf{Класс} & \textbf{Количество} & \textbf{Доля (\%)} & \textbf{Характеристика} \\
\hline
miner & 3,547 & 43.7 & Майнеры и майнинг-пулы \\
exchange & 1,532 & 18.9 & Криптовалютные биржи \\
services & 1,249 & 15.4 & Финансовые сервисы \\
gambling & 911 & 11.2 & Азартные игры \\
coinjoin-like & 787 & 9.7 & Микшинг-сервисы \\
mining\_pool & 89 & 1.1 & Пулы майнинга \\
\hline
\textbf{Всего} & \textbf{8,115} & \textbf{100.0} & \\
\hline
\end{tabular}
\caption{Распределение классов в датасете}
\label{tab:class_distribution}
\end{table}

\subsubsection{Корреляционный анализ признаков}

Корреляционный анализ основных признаков выявил следующие значимые взаимосвязи:

\begin{itemize}
    \item \textbf{Сильная положительная корреляция} (r > 0.7):
        \begin{itemize}
            \item \texttt{avg\_amount\_sent} и \texttt{avg\_amount\_received} (r = 0.85)
            \item \texttt{txs\_count} и \texttt{time\_span\_days} (r = 0.78)
        \end{itemize}
    \item \textbf{Умеренная положительная корреляция} (0.3 < r < 0.7):
        \begin{itemize}
            \item \texttt{avg\_inputs} и \texttt{avg\_outputs} (r = 0.52)
            \item \texttt{avg\_tx\_size} и \texttt{avg\_inputs} (r = 0.45)
        \end{itemize}
    \item \textbf{Отрицательная корреляция}:
        \begin{itemize}
            \item \texttt{dust\_ratio} и \texttt{avg\_amount\_sent} (r = -0.34)
            \item \texttt{transaction\_frequency} и \texttt{time\_span\_days} (r = -0.28)
        \end{itemize}
\end{itemize}

\subsubsection{Сравнительный анализ по классам}

Статистический анализ показал характерные различия между классами:

\begin{table}[h]
\centering
\begin{tabular}{|l|c|c|c|c|}
\hline
\textbf{Класс} & \textbf{txs\_count} & \textbf{avg\_amount\_sent} & \textbf{transaction\_frequency} & \textbf{dust\_ratio} \\
& \textbf{(медиана)} & \textbf{(медиана, BTC)} & \textbf{(медиана)} & \textbf{(медиана)} \\
\hline
miner & 9.0 & 0.0181 & 0.0485 & 0.12 \\
exchange & 17.0 & 0.0422 & 0.0690 & 0.08 \\
services & 4.0 & 0.0010 & 0.0524 & 0.15 \\
gambling & 84.0 & 0.0122 & 0.6786 & 0.22 \\
coinjoin-like & 55.0 & 0.1000 & 0.2168 & 0.18 \\
mining\_pool & 99.5 & 10.2279 & 1.1497 & 0.05 \\
\hline
\end{tabular}
\caption{Ключевые статистики по классам}
\label{tab:class_statistics}
\end{table}

\textbf{Ключевые наблюдения}:
\begin{itemize}
    \item \textbf{mining\_pool} демонстрирует наибольшие суммы транзакций (медиана 10.23 BTC)
    \item \textbf{gambling} характеризуется высокой частотой транзакций (0.68 транзакций/день)
    \item \textbf{services} показывают наименьшую активность (4 транзакции, медиана)
    \item \textbf{coinjoin-like} имеют средние значения, отражающие смешанную активность
\end{itemize}

\subsubsection{Визуализация распределений признаков}

Для анализа данных были построены следующие типы визуализаций:

\begin{itemize}
    \item \textbf{Box-plot диаграммы} для сравнения распределений признаков по классам
    \item \textbf{Корреляционная матрица} с тепловой картой для выявления взаимосвязей
    \item \textbf{Гистограммы} распределения классов
    \item \textbf{Логарифмические шкалы} для экономических признаков с большим разбросом значений
\end{itemize}

\textbf{Примечание}: Полный код анализа данных и визуализации представлен в Jupyter notebook \texttt{bitcoin\_analysis.ipynb} (см. приложение А). Notebook содержит интерактивные графики, статистический анализ и выводы по каждому типу признаков.

Визуализация показала, что многие признаки имеют логнормальное распределение, что требует применения логарифмических преобразований для нормализации данных.

\subsection{Технические детали реализации}

\subsubsection{Архитектура системы извлечения признаков}

Система извлечения признаков реализована на языке Python с использованием следующих технологий:

\begin{itemize}
    \item \textbf{pandas} - для обработки и анализа данных
    \item \textbf{numpy} - для векторизованных вычислений
    \item \textbf{matplotlib/seaborn} - для визуализации
    \item \textbf{scikit-learn} - для предобработки данных
\end{itemize}

\textbf{Примечание}: Полный исходный код системы извлечения признаков представлен в приложении А данного отчета. Код включает класс \texttt{BitcoinTransactionAnalyzer} с методами для извлечения всех 34 признаков, а также Jupyter notebook с анализом данных и визуализацией результатов.

\subsubsection{Оптимизация производительности}

Для обработки больших объемов данных применены следующие оптимизации:

\begin{enumerate}
    \item \textbf{Векторизованные операции NumPy} - замена циклов на матричные операции
    \item \textbf{Инкрементальное сохранение} - сохранение результатов каждые 50 обработанных адресов
    \item \textbf{Пакетная обработка} - группировка операций для минимизации I/O
    \item \textbf{Память-эффективные структуры данных} - использование категориальных типов pandas
\end{enumerate}

\subsubsection{Обработка выбросов и нормализация}

\begin{itemize}
    \item \textbf{Логарифмические преобразования} для признаков с логнормальным распределением
    \item \textbf{Обрезка выбросов} на уровне 99-го процентиля
    \item \textbf{Стандартизация} признаков с использованием Z-score нормализации
    \item \textbf{Обработка пропущенных значений} методом медианной импутации
\end{itemize}

\subsection{Выводы по обработке данных}

Проведенный комплексный анализ показал:

\begin{enumerate}
    \item \textbf{Несбалансированность датасета}: доминирование класса \texttt{miner} (43.7\%) требует применения техник балансировки классов
    \item \textbf{Четкие различия между классами}: структурные и экономические признаки демонстрируют статистически значимые различия
    \item \textbf{Корреляционные взаимосвязи}: подтверждены логические связи между признаками, что позволяет использовать методы снижения размерности
    \item \textbf{Характерные паттерны}: каждый класс имеет уникальные поведенческие особенности, выявленные через визуализацию
    \item \textbf{Техническая готовность}: система извлечения признаков обеспечивает воспроизводимость и масштабируемость
\end{enumerate}

\subsubsection{Рекомендации для дальнейшей работы}

\begin{itemize}
    \item Применение техник балансировки классов (SMOTE, undersampling)
    \item Использование методов отбора признаков для снижения размерности
    \item Валидация на временных данных для проверки стабильности признаков
    \item Анализ важности признаков для интерпретируемости модели
\end{itemize}

Полученные результаты обработки данных создают надежную основу для построения высокоточных моделей классификации Bitcoin-адресов и обеспечивают научную обоснованность выбора признаков.

\section{Приложение А. Исходный код}

\subsection{Скрипт извлечения признаков}

Полный исходный код класса \texttt{BitcoinTransactionAnalyzer} для извлечения признаков из транзакционных данных Bitcoin:

\lstinputlisting[language=Python, caption={Класс BitcoinTransactionAnalyzer для извлечения признаков}, label={lst:bitcoin_analyzer}]{bitcoin_analyzer_code.py}

\subsection{Jupyter Notebook для анализа данных}

Основные блоки кода Jupyter Notebook для анализа и визуализации признаков Bitcoin-адресов:

\lstinputlisting[language=Python, caption={Основные блоки кода анализа признаков Bitcoin-адресов}, label={lst:bitcoin_notebook}]{bitcoin_analysis_code.py}
