\abstract % Структурный элемент: РЕФЕРАТ
\textbf{Цель работы:} разработка комплексной методологии подготовки датасета для обучения классификатора типов Bitcoin-кошельков, включающей модули сбора данных через Bitcoin-эксплорер и извлечения признаков из транзакционных данных.

\textbf{Задачи исследования:} анализ существующих источников данных Bitcoin-адресов и их категоризации, разработка системы автоматизированного сбора данных через API Bitcoin-эксплорера, создание методологии извлечения 34 признаков из транзакционных данных, разработка алгоритмов анализа структурных, временных, экономических и сетевых паттернов, проведение комплексного статистического анализа извлеченных признаков, создание системы валидации и контроля качества данных.

\textbf{Объект исследования:} Bitcoin-адреса и связанные с ними транзакции в сети Bitcoin.

\textbf{Предмет исследования:} методы сбора, обработки и анализа данных о Bitcoin-адресах для создания качественных датасетов машинного обучения с целью классификации типов кошельков.

\textbf{Методы исследования:} анализ существующих датасетов Bitcoin-адресов, разработка алгоритмов сбора данных через REST API Bitcoin-эксплорера, методы извлечения признаков из транзакционных данных, статистический анализ качества данных, корреляционный анализ, методы машинного обучения для оценки важности признаков.

\textbf{Результаты:} разработана многоуровневая архитектура системы сбора данных с использованием API Bitcoin-эксплорера, создана методология извлечения 34 признаков, разделенных на 6 категорий (структурные, UTXO паттерны, временные, экономические, сетевые, поведенческие), проведен комплексный статистический анализ датасета из 8,115 Bitcoin-адресов с выявлением наиболее информативных признаков для классификации, создана система валидации и контроля качества данных с обработкой ошибок и параллельной обработкой.

\textbf{Практическая значимость:} результаты исследования могут быть использованы для создания качественных датасетов для обучения алгоритмов классификации типов Bitcoin-кошельков, автоматизации процессов подготовки данных для машинного обучения в области Bitcoin-аналитики, разработки инструментов для регуляторного надзора и анализа рисков в криптовалютных операциях, а также для исследования экономических паттернов в экосистеме Bitcoin.

\textbf{Ключевые слова:} Bitcoin, блокчейн, машинное обучение, классификация кошельков, датасет, извлечение признаков, Bitcoin-эксплорер, транзакционный анализ.