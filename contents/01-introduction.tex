\introduction % Структурный элемент: ВВЕДЕНИЕ

С момента появления Bitcoin в 2009 году эта криптовалюта стала неотъемлемой частью современной финансовой экосистемы. Децентрализованная природа Bitcoin, основанная на технологии блокчейн, обеспечивает прозрачность и безопасность транзакций, но одновременно создает возможности для злоупотреблений, таких как отмывание денег, мошенничество, фишинг и финансирование незаконной деятельности. В отличие от традиционных банковских систем, где информация о финансовых операциях доступна ограниченному кругу лиц, Bitcoin предоставляет открытый доступ к данным о транзакциях. Это делает его привлекательным для анализа, но также требует разработки эффективных алгоритмов классификации типов кошельков для выявления подозрительной активности в сети Bitcoin.

Классификация Bitcoin-адресов по типам кошельков является критически важной задачей в области анализа блокчейн-данных. Точная идентификация типов кошельков (биржи, майнеры, микшинг-сервисы, азартные игры) имеет практическое значение для регуляторного надзора, анализа рисков в криптовалютных операциях, исследования экономических паттернов в экосистеме Bitcoin и разработки инструментов для правоохранительных органов. Проблема классификации Bitcoin-адресов приобретает особую актуальность в связи с ростом объемов криптовалютных транзакций (ежедневный объем Bitcoin превышает 300,000 операций), регуляторными требованиями соблюдения AML/KYC процедур, необходимостью выявления подозрительной активности и мошеннических схем, а также исследовательскими задачами понимания экономических паттернов в криптовалютной экосистеме.

Ключевой проблемой в области анализа Bitcoin-транзакций является отсутствие качественных, сбалансированных и репрезентативных датасетов для обучения классификаторов типов кошельков. Существующие датасеты часто страдают от дисбаланса классов, неполноты данных и устаревших меток, что существенно ограничивает эффективность алгоритмов машинного обучения. Кроме того, быстрая эволюция сети Bitcoin и появление новых типов сервисов требуют постоянного обновления и расширения обучающих данных.

Целью данной научно-исследовательской работы является разработка комплексной методологии подготовки датасета для обучения классификатора типов Bitcoin-кошельков, включающей модули сбора данных через Bitcoin-эксплорер и извлечения признаков из транзакционных данных.

Основные задачи исследования:
\begin{itemize}
    \item Анализ существующих источников данных Bitcoin-адресов и их категоризации
    \item Разработка системы автоматизированного сбора данных через API Bitcoin-эксплорера
    \item Создание методологии извлечения 34 признаков из транзакционных данных
    \item Разработка алгоритмов анализа структурных, временных, экономических и сетевых паттернов
    \item Проведение комплексного статистического анализа извлеченных признаков
    \item Создание системы валидации и контроля качества данных
\end{itemize}

Актуальность работы обусловлена растущей потребностью в качественных датасетах для обучения алгоритмов классификации типов Bitcoin-кошельков, необходимостью автоматизации процессов подготовки данных для машинного обучения в области Bitcoin-аналитики, а также важностью понимания поведенческих паттернов различных типов участников криптовалютной экосистемы для регуляторного надзора и анализа рисков.