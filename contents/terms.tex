% Термины и определения
\newglossaryentry{bitcoin}{
    name={Bitcoin},
    description={Криптовалюта и одноименная платежная система, использующая одноименную единицу для учета операций}
}

\newglossaryentry{blockchain}{
    name={Блокчейн},
    description={Распределенная база данных, состоящая из последовательно связанных блоков, содержащих информацию о транзакциях}
}

\newglossaryentry{transaction}{
    name={Транзакция},
    description={Операция передачи Bitcoin между адресами, записанная в блокчейне}
}

\newglossaryentry{address}{
    name={Адрес},
    description={Уникальный идентификатор в сети Bitcoin, используемый для отправки и получения средств}
}

\newglossaryentry{utxo}{
    name={UTXO},
    description={Unspent Transaction Output - модель учета средств в Bitcoin, где каждый выход транзакции может быть потрачен только один раз}
}

\newglossaryentry{wallet}{
    name={Кошелек},
    description={Программное обеспечение или устройство для хранения, отправки и получения Bitcoin}
}

\newglossaryentry{mining}{
    name={Майнинг},
    description={Процесс создания новых блоков в блокчейне Bitcoin путем решения криптографических задач}
}

\newglossaryentry{hash}{
    name={Хеш},
    description={Результат применения криптографической хеш-функции к данным, используется для обеспечения целостности и безопасности}
}

\newglossaryentry{private_key}{
    name={Приватный ключ},
    description={Секретный ключ, используемый для подписи транзакций и доступа к средствам на Bitcoin-адресе}
}

\newglossaryentry{public_key}{
    name={Публичный ключ},
    description={Криптографический ключ, доступный всем участникам сети, используется для проверки подписей транзакций}
}

\newglossaryentry{block}{
    name={Блок},
    description={Структура данных, содержащая набор транзакций и метаданные, является частью блокчейна}
}

\newglossaryentry{node}{
    name={Узел},
    description={Компьютер в сети Bitcoin, который поддерживает копию блокчейна и участвует в валидации транзакций}
}

\newglossaryentry{consensus}{
    name={Консенсус},
    description={Механизм достижения согласия между узлами сети о состоянии блокчейна}
}

\newglossaryentry{proof_of_work}{
    name={Proof of Work},
    description={Алгоритм консенсуса, требующий от майнеров решения вычислительно сложных задач для создания блоков}
}

\newglossaryentry{merkle_tree}{
    name={Дерево Меркла},
    description={Структура данных, используемая для эффективной проверки целостности большого количества транзакций в блоке}
}

\newglossaryentry{script}{
    name={Script},
    description={Язык программирования Bitcoin, используемый для определения условий траты средств}
}

\newglossaryentry{multisig}{
    name={Мультиподпись},
    description={Механизм, требующий подписи нескольких приватных ключей для выполнения транзакции}
}

\newglossaryentry{segwit}{
    name={SegWit},
    description={Segregated Witness - обновление протокола Bitcoin, отделяющее данные подписей от данных транзакций}
}

\newglossaryentry{lightning_network}{
    name={Lightning Network},
    description={Протокол второго уровня для Bitcoin, обеспечивающий быстрые и дешевые микроплатежи}
}

\newglossaryentry{coinjoin}{
    name={CoinJoin},
    description={Метод повышения приватности в Bitcoin, объединяющий входы и выходы нескольких пользователей в одну транзакцию}
}

\newglossaryentry{coinbase_tx}{
    name={Coinbase транзакция},
    description={Специальная транзакция, создаваемая майнером в каждом блоке для получения вознаграждения за майнинг}
}

\newglossaryentry{feature_extraction}{
    name={Извлечение признаков},
    description={Процесс выделения характеристик из данных для использования в алгоритмах машинного обучения}
}

\newglossaryentry{classification}{
    name={Классификация},
    description={Задача машинного обучения, заключающаяся в отнесении объектов к предопределенным классам}
}

\newglossaryentry{dataset}{
    name={Датасет},
    description={Набор данных, используемый для обучения, тестирования и валидации алгоритмов машинного обучения}
}

\newglossaryentry{api_client}{
    name={API клиент},
    description={Программный компонент, обеспечивающий взаимодействие с внешним API для получения данных}
}

\newglossaryentry{rate_limiting}{
    name={Ограничение скорости},
    description={Механизм контроля частоты запросов к API для предотвращения перегрузки сервера}
}

\newglossaryentry{parallel_processing}{
    name={Параллельная обработка},
    description={Метод выполнения вычислений одновременно на нескольких процессорах или потоках}
}

\newglossaryentry{data_persistence}{
    name={Персистентность данных},
    description={Способность данных сохраняться после завершения работы программы}
}

\newglossaryentry{incremental_collection}{
    name={Инкрементальный сбор},
    description={Метод сбора данных, при котором обрабатываются только новые или измененные данные}
}
