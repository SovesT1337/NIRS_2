\section{Заключение}

В рамках данной научно-исследовательской работы был разработан метод создания единого адресного датасета Bitcoin с признаками, извлеченными через Bitcoin-эксплорер, и экспертной разметкой из существующих источников.

\subsection{Достигнутые результаты}

\begin{enumerate}
    \item Проведен анализ существующих Bitcoin-датасетов и выявлены их ключевые характеристики
    \item Разработан метод извлечения адресов и сбора их признаков через Bitcoin-эксплорер
    \item Предложена система унификации разметки на основе экспертных данных
    \item Создана методология разрешения конфликтов при пересечении адресов в разных датасетах
    \item Обоснован подход создания единого адресного датасета как наиболее практичного решения
\end{enumerate}

\subsection{Практическая значимость}

Результаты исследования имеют высокую практическую значимость для:

\begin{itemize}
    \item Исследователей в области анализа Bitcoin-транзакций
    \item Разработчиков систем обнаружения мошенничества в криптовалютах
    \item Аналитиков блокчейн-безопасности
    \item Правоохранительных органов для расследования преступлений с Bitcoin
\end{itemize}

\subsection{Направления дальнейших исследований}

Перспективными направлениями дальнейших исследований являются:

\begin{itemize}
    \item Практическая реализация метода создания единого адресного датасета
    \item Разработка алгоритмов сбора признаков адресов через Bitcoin-эксплорер
    \item Создание системы валидации качества разметки и разрешения конфликтов
    \item Разработка методов машинного обучения для классификации адресов
    \item Интеграция дополнительных источников экспертной разметки
    \item Создание системы мониторинга новых мошеннических адресов
\end{itemize}
