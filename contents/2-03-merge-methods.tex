\subsection{Метод создания единого адресного датасета}

Для создания единого датасета Bitcoin-адресов предлагается следующий подход:

\subsubsection{Извлечение адресов из существующих датасетов}

Из каждого доступного датасета извлекаются списки адресов с их разметкой:

\begin{itemize}
    \item \textbf{Bitcoin Heist}: 2,916,697 адресов с метками ransomware/легитимные
    \item \textbf{Real-CATS}: 209,399 адресов с метками криминальные/биржевые
    \item \textbf{BABD}: 544,462 адреса с метками типов адресов
    \item \textbf{Elliptic}: Адреса из транзакций с агрегированными характеристиками
\end{itemize}

\subsubsection{Сбор признаков адресов через Bitcoin-эксплорер}

Для каждого адреса собираются следующие типы признаков через API Bitcoin-эксплорера:

\begin{itemize}
    \item \textbf{Базовые характеристики}: Баланс, количество входящих/исходящих транзакций, первый/последний блок
    \item \textbf{Топологические признаки}: Количество уникальных связей, степень центральности, кластеризационные коэффициенты
    \item \textbf{Поведенческие признаки}: Частота транзакций, средние объемы, паттерны активности
    \item \textbf{Временные признаки}: Период активности, интервалы между транзакциями, сезонность
    \item \textbf{Статистические признаки}: Дисперсия объемов, экстремальные значения, тренды
    \item \textbf{Графовые признаки}: Метрики связности, центральность по различным мерам
\end{itemize}

\subsubsection{Унификация разметки}

Создается единая система меток на основе экспертной разметки из исходных датасетов:

\begin{itemize}
    \item \textbf{Бинарная классификация}: Легитимные vs Подозрительные адреса
    \item \textbf{Тип мошенничества}: Ransomware, Darknet, Scam, Money Laundering, Other
    \item \textbf{Уровень достоверности}: Высокая, Средняя, Низкая (на основе количества источников)
\end{itemize}

\subsubsection{Разрешение конфликтов}

При наличии адреса в нескольких датасетах с разной разметкой применяются следующие правила:

\begin{itemize}
    \item \textbf{Приоритет источников}: Real-CATS (реальные дела) > Bitcoin Heist > BABD > Elliptic
    \item \textbf{Консенсус}: Если большинство источников указывают на мошенничество
    \item \textbf{Консервативный подход}: При сомнениях адрес помечается как подозрительный
\end{itemize}
