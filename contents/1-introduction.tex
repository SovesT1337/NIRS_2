\introduction % Структурный элемент: ВВЕДЕНИЕ

С момента появления Bitcoin в 2009 году эта криптовалюта стала неотъемлемой частью современной финансовой экосистемы. Децентрализованная природа Bitcoin, основанная на технологии блокчейн, обеспечивает прозрачность и безопасность транзакций, но одновременно создает возможности для злоупотреблений, таких как отмывание денег, мошенничество, фишинг и финансирование незаконной деятельности. В отличие от традиционных банковских систем, где информация о финансовых операциях доступна ограниченному кругу лиц, Bitcoin предоставляет открытый доступ к данным о транзакциях. Это делает его привлекательным для анализа, но также требует разработки эффективных алгоритмов классификации транзакций для выявления подозрительной активности в сети Bitcoin.

Ключевой проблемой в области анализа Bitcoin-транзакций является отсутствие качественных, сбалансированных и репрезентативных датасетов для обучения классификаторов. Существующие датасеты часто страдают от дисбаланса классов, неполноты данных и устаревших меток, что существенно ограничивает эффективность алгоритмов машинного обучения. Кроме того, быстрая эволюция сети Bitcoin и появление новых типов мошеннических схем требуют постоянного обновления и расширения обучающих данных.

Целью данной научно-исследовательской работы является разработка комплексной системы подготовки датасета для обучения классификатора Bitcoin-транзакций, включающей модули сбора данных и дата инжиниринга.

Основные задачи исследования:
\begin{itemize}
    \item Анализ существующих источников данных Bitcoin-транзакций
    \item Разработка системы автоматизированного сбора данных из различных источников
    \item Создание методов предобработки и очистки данных
    \item Разработка алгоритмов извлечения признаков и построения графов Bitcoin-транзакций
    \item Решение проблемы дисбаланса классов в датасете
    \item Создание системы валидации и контроля качества данных
\end{itemize}

Актуальность работы обусловлена растущей потребностью в качественных датасетах для обучения алгоритмов обнаружения мошеннических транзакций в сети Bitcoin, а также необходимостью автоматизации процессов подготовки данных для машинного обучения в области Bitcoin-аналитики.
